\section{Model parameter description}

\subsection{Introduction}

The simulation model parameter descriptions below consists of several parts, described here in detail:

\textbf{MC model parameter:} \textit{Descriptive name of model parameter}

\textbf{Description:} \textit{short description of model parameter; allows to understand its functionality}

\textbf{Format and units:} \textit{precise description of expected data formats and units}

\textbf{Required accuracy:} \textit{Guidance on expected accuracy (e.g., the sampling speed of the readout system might be given exactly, while other values are expected to be given to 1\% accuracy)}
\TODO{Discuss this: overlap with calibration and data provided by the instrument systems}

\textbf{Setting procedure:}
Short description and list of relevant setting / derivation groups

\textbf{Validation groups:} \textit{Short description and list of relevant validation groups}

\textbf{Systematic uncertainty estimation:}
list of relevant systematic uncertainty estimation procedures; bracketing curves (energy dependence...) 

\textbf{Update frequency and averaging:}
e.g., averaging over time, or assume that all elements are the same, etc

\textbf{Data source:}
ACE, laboratory, etc. (needed)?

\textbf{Notes:}
 any further description / notes required



%%%%%%%%%%%%%%%%%%%%%%%%%%%%%%%%%%%%%%%%%%%%%%%%%%%%%%%%%%%%%%%%%%%%%%%%%%%%%%%%%%%%%%%%%%%%%%%%%%%
\subsection{Site Parameters}

\subsubsection{Molecular profiles}

\textbf{MC model parameter: atmospheric profile}

\textbf{Description: }
Atmospheric model profiles with molecular density, atmospheric thickness, and index of refraction as function of altitude.

\textbf{Format and units:}

Table format with the following columns:

\begin{itemize}
\item altitude: (unit: km) physical altitude above sea level in the range [0, 120] km; step size of 1 km in the altitude range [0,30] km, steps of 2 km for [30,50] km, steps of 5 km above.

\item density: (unit g/cm$^3$) atmospheric density

\item index of refraction: (unit: none) index of refraction calculated at at fixed wavelength (typically at 400 nm)

\end{itemize}

\TODO{Further columns are usually given in the atmprof files: thickness, temperature, pressure, partival vapor pressure; check if they are needed}

\TODO{Are altitude steps fixed?}

\textbf{Required accuracy:}

\textbf{Setting procedure:}
Short description and list of relevant setting procedure

\textbf{Setting groups:}

\begin{enumerate}
    \item \nameref{settingGroup:AtmosphericProperties}
\end{enumerate}

\textbf{Validation groups:}

\begin{enumerate}
    \item \nameref{validationGroup:ModelParameterComparison}
    \begin{enumerate}
         \item evaluate density vs altitude
         \item evaluate of index of refraction vs altitude
    \end{enumerate}
    \item \nameref{validationGroup:ReferenceModelComparison}
    \item \nameref{validationGroup:DataMonteCarloComparison}
\end{enumerate}

Short description and list of relevant validation groups 

\textbf{Systematic uncertainty estimation:}
Generation of bracketing IRFs (\nameref{systematics:IRFmetric}) using the following input:
\begin{enumerate}
    \item min / max atmospheric profiles obtained from the atmospheric profile generator; calculation of differences relative to model parameters under consideration
\end{enumerate}

\textbf{Update frequency and averaging:}
e.g., averaging over time, or assume that all elements are the same, etc

\textbf{Data source:}
ACE, laboratory, etc. (needed)?

\textbf{Notes:}
 any further description / notes required

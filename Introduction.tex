\section{Introduction}

This document describes the CTA MC model parameters, the corresponding workflows for setting and validating them, and the procedures to calculate the corresponding systematic uncertainties.

This document is part of a series describing the CTA MC Simulation Pipeline. 
It is expected that the reader is familiar with the wider concept of \simsys as described in  \cite{CTAConcept}.

%%%%%%%%%%%%%%%%%%%%%%%%%%%

\section{Glossary}

Useful definitions helpful for this document are listed below.
We point also to the glossary definitions in Jama and those in  \cite{CTAConcept}.

\begin{description}

\item[Reference simulation model:] 
Simulation model defined as reference for comparisons with newly introduced model parameters.
Any simulation model can be defined as a reference model.

\item[Reference simulation production: ]
Simulation production based on reference simulation model. 

\item[Reference Instrument Response Functions: ]  
Set of instrument response functions (IRFs) calculated from the reference simulation production used for comparisons with IRFs derived from a simulation production with an altered simulation model.
Typical instrument response functions are effective areas, reconstructed energy, and gamma-ray point-spread function.

\item[Array Common Elements (ACE): ]

\end{description}



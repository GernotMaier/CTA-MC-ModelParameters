\section{Introduction}

This document describes the CTA MC model parameters, the corresponding workflows for setting and validating them, and the procedures to calculate the corresponding systematic uncertainties.

This document is part of a series describing the CTA MC Simulation Pipeline:

\begin{itemize}
    \item Concept for the CTA Simulation Pipelines System \cite{CTAConcept}
    \item Simulation Model ParametersDescription and Workflows (this document)
    \item Prototype implementations for the CTA Simulation Pipelines System
    \item Data model and formats for the CTA Simulation Pipelines System 
\end{itemize}
It is expected that the reader is familiar with the wider concept of \simsys as described in  \cite{CTAConcept}.

\textbf{Summary:}
\begin{enumerate}
    \item Simulation model parameters are described in the Section \textbf{\nameref{sect:ModelParameterDescription}} (Section \ref{sect:ModelParameterDescription}). 
    This includes the parameter format and units, the required accuracy, parameter setting and validation procedures, and update frequencies.
    
    \item The Section \textbf{\nameref{sect:ModelParameterSetting}} (Section\ref{sect:ModelParameterSetting}) describes the groups of parameters which are set or derived simultaneously. Each model parameter is a member of at least one model parameter setting group.
    
    \item Each model parameter is part of at least one parameter validation group and is validated in at least one validation workflows as described in the section \textbf{\nameref{sect:ModelParameterValidation}} (Section \ref{sect:ModelParameterValidation}).
    
    \item  \textbf{\nameref{sect:Systematics}} (Section \ref{sect:Systematics}) allow to derive for each parameter contributing to the systematic uncertainties this quantity on a regular basis.
    
    \item The workflows described above require different \textbf{\nameref{sect:SimulationTypes}} (Section \ref{sect:SimulationTypes})
    
\end{enumerate}

%%%%%%%%%%%%%%%%%%%%%%%%%%%

\section{Glossary}

Useful definitions helpful for this document are listed below.
We point also to the glossary definitions in Jama and those in  \cite{CTAConcept}.

\begin{description}

\item[Reference simulation model:] 
Simulation model defined as reference for comparisons with newly introduced model parameters.
Any simulation model can be defined as a reference model.

\item[Reference simulation production: ]
Simulation production based on reference simulation model. 

\item[Reference Instrument Response Functions: ]  
Set of instrument response functions (IRFs) calculated from the reference simulation production used for comparisons with IRFs derived from a simulation production with an altered simulation model.
Typical instrument response functions are effective areas, reconstructed energy, and gamma-ray point-spread function.

\item[Array Common Elements (ACE): ]

\end{description}


